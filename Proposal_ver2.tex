\documentclass[11pt,a4paper]{article}

% for Chinese
%\usepackage{fontspec}  % 加這個就可以設定字體
%\usepackage[BoldFont, SlantFont]{xeCJK}  % 讓中英文字體分開設置
%\setCJKmainfont{新細明體}  % 設定中文為系統上的字型,而英文不去更動,使用原TeX\字型

% useful packages
\usepackage{amsfonts}
\usepackage{amssymb}
\usepackage{amsmath}
\usepackage{amsthm}
\usepackage{epsfig}
\usepackage{graphicx}
\usepackage{natbib}
\usepackage{textcomp}
\usepackage{booktabs}
\usepackage{multirow}
\usepackage{url}
\usepackage{color}
\usepackage{fullpage}
\usepackage[capitalize]{cleveref}
\usepackage{mathtools}
\usepackage{enumitem}
\usepackage{authblk}
\usepackage{algorithm}
\usepackage{algpseudocode}
\usepackage{subcaption}
\usepackage{tikz}
\usepackage{tabularx}
\usepackage{adjustbox}
\usepackage{pgfplots}
% basic setting
\renewcommand{\baselinestretch}{1.25}
\parskip=5pt
\parindent=20pt
\footnotesep=5mm

% abbreviation
\newtheorem{lem}{Lemma}
\newtheorem{prop}{Proposition}
\newtheorem{thm}{Theorem}
\newtheorem{defn}{Definition}
\newtheorem{cor}{Corollary}
\newtheorem{assp}{Assumption}
\newtheorem{obs}{Observation}
\newenvironment{pf}{\begin{proof}\vspace{-10pt}}{\end{proof}}
% \newtheorem{ques}{Question}
% \newtheorem{rmk}{Remark}
% \newtheorem{note}{Note}
% \newtheorem{eg}{Example}

\newenvironment{enumerateTight}{\begin{enumerate}\vspace{-8pt}}{\end{enumerate}\vspace{-8pt}}
\newenvironment{itemizeTight}{\begin{itemize}\vspace{-8pt}}{\end{itemize}\vspace{-8pt}}
\leftmargini=25pt   % default: 25pt
\leftmarginii=12pt  % default: 22pt
\pgfplotsset{compat=1.17}

\title{Operations Research, Spring 2025 (113-2) \\ Final Project Proposal}

\author{Zi-Yi Jau} % b12705064
\author{Bing-Zhe Wu} % b12705049
\author{Chung-Kai Lin} % b12705052
\author{Zhi-Xin Lin} % b12705013
\author{Nan-Tien Lai} % b12705010
\affil{Department of Information Management, National Taiwan University}



\begin{document}

\maketitle

\section{Introduction}

Shared bicycle systems have become a vital component of urban mobility solutions, providing low-carbon, convenient, and short-distance transportation alternatives. However, operational inefficiencies, such as bike unavailability or full docking stations, have led to increased user waiting times and decreased service satisfaction. These inefficiencies are exacerbated by the inability of existing data systems to fully capture unfulfilled demands—users who abandon stations due to lack of available bikes or docking spaces. In response, municipal governments and system operators, such as Taipei City's YouBike managers, face challenging decisions to balance user satisfaction, operational costs, and logistical complexity. Various dispatching strategies have been explored to optimize service quality, including dynamic redistribution, capacity adjustments, and temporary parking (or ``bike hiding'') areas. This study aims to address these operational challenges by developing an optimization model to improve user experience and system efficiency specifically for Taipei City's YouBike system.

\section{Problem description}

We aim to address three main operational challenges faced by Taipei City's YouBike system operators:
\begin{itemize}
    \item \textbf{Bike availability}: ensuring sufficient bikes for users who want to rent.
    \item \textbf{Return success rate}: avoiding full stations where users cannot return bikes.
    \item \textbf{User waiting time}: minimizing the time users spend waiting due to bike unavailability or full docks.
\end{itemize}

Traditional dispatch operations often involve transporting bikes between stations using trucks. However, this can result in redundant movements—for example, bikes removed in the morning are often returned to the same station later in the day. Moreover, existing data fail to fully capture unobserved demand, which includes potential users leaving due to bike shortages or full stations, leading decision-makers to underestimate actual user dissatisfaction. Recent insights from Taipei’s pilot test in Shilin and Xinyi districts suggest that improved station capacity utilization and localized temporary storage ("bike hiding") strategies could reduce the need for frequent inter-station redistributions, effectively balancing user satisfaction and operational efficiency.

\section{Mathematical model}

Let $I$ denote the set of stations. For each station $i \in I$, define the following parameters and decision variables:

\begin{itemize}
    \item Parameters:
    \begin{itemize}
        \item $B_i^0$: Initial number of bikes at station $i$
        \item $C_i$: Total docking capacity at station $i$
        \item $D_i^{\text{borrow}}, D_i^{\text{return}}$: Forecasted observed demand for borrowing and returning bikes
        \item $D_i^{\text{borrow, true}}, D_i^{\text{return, true}}$: Actual demand (including unobserved demand) estimated statistically
        \item $\mu$: Average service rate (number of bikes rented or returned per minute)
        \item $\alpha, \beta$: Cost coefficients reflecting redistribution and hiding operational costs
        \item $K$: Maximum number of bikes allowed to be redistributed system-wide
        \item $H$: Maximum number of bikes allowed for hiding operations system-wide
    \end{itemize}

    \item Decision variables:
    \begin{itemize}
        \item $x_{ij}$: Number of bikes redistributed from station $i$ to station $j$
        \item $h_i^{\text{in}}, h_i^{\text{out}}$: Number of bikes temporarily hidden into or removed from local temporary storage at station $i$
        \item $W_i^{\text{borrow}}, W_i^{\text{return}}$: Expected waiting times at station $i$ for borrowing and returning bikes
    \end{itemize}
\end{itemize}

The bike balance equation at station $i$ is given by:
\[
B_i = B_i^0 + \sum_{j \ne i} x_{ji} - \sum_{j \ne i} x_{ij} + h_i^{\text{in}} - h_i^{\text{out}}
\]

Waiting time is estimated as:
\[
W_i^{\text{borrow}} = \max\left(0, \frac{D_i^{\text{borrow, true}} - B_i}{\mu}\right), \quad
W_i^{\text{return}} = \max\left(0, \frac{D_i^{\text{return, true}} - (C_i - B_i)}{\mu}\right)
\]

Our objective is to minimize total user waiting time and operational costs, balancing the trade-off between user satisfaction and system efficiency. Therefore, the objective function is:
\[
\min \sum_{i \in I} \left( W_i^{\text{borrow}} + W_i^{\text{return}} \right) + \alpha \sum_{i,j} x_{ij} + \beta \sum_{i} \left( h_i^{\text{in}} + h_i^{\text{out}} \right)
\]

subject to:
\begin{align*}
& B_i \leq C_i &\quad &\text{(capacity constraint for each station)} \\
& x_{ij}, h_i^{\text{in}}, h_i^{\text{out}} \geq 0 &\quad &\text{(non-negativity constraints)} \\
& \sum_{i,j} x_{ij} \leq K &\quad &\text{(system-wide redistribution limit)} \\
& \sum_{i}(h_i^{\text{in}} + h_i^{\text{out}}) \leq H &\quad &\text{(system-wide hiding limit)}
\end{align*}

Note: Actual demand ($D_i^{\text{borrow, true}}$, $D_i^{\text{return, true}}$) will be estimated based on historical data analysis and field observations to capture users who abandon stations due to unavailability.

\section{Expected results}

To evaluate the potential impact of the proposed optimization model, we plan to simulate a representative subset of Taipei City’s YouBike system. This initial simulation will include several stations with varying characteristics, such as different initial bike counts, docking capacities, and estimated true demands (including unobserved demands).

We will compare the following two strategies:

\begin{enumerate}
    \item \textbf{Traditional redistribution only}: Bikes are redistributed exclusively by trucks moving between stations, without any temporary local storage.
    \item \textbf{Optimized strategy (proposed model)}: Bikes are redistributed strategically, using localized temporary storage ("bike hiding") to absorb short-term demand fluctuations and minimize redundant bike transfers.
\end{enumerate}

We expect the optimized strategy to demonstrate a significant reduction in user waiting times and operational redundancies compared to the traditional strategy. Particularly, we anticipate substantial improvements at stations with historically high demand-supply imbalances. 

Furthermore, our analysis aims to quantify the effectiveness of localized hiding operations in mitigating unobserved demand surges, thus reducing the overall need for costly inter-station redistributions. This simulation, based on realistic assumptions derived from historical data and field observations, will provide practical insights and policy recommendations for YouBike system operators and municipal transportation authorities.
\vspace{0.5em}


\end{document}
